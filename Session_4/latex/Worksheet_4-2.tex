\documentclass[11pt,letterpaper]{article}
\usepackage[utf8]{inputenc}
\usepackage[left=1in,right=1in,top=1in,bottom=1in]{geometry}
\usepackage{amsfonts,amsmath}
\usepackage{graphicx,float}
% -----------------------------------
\usepackage{hyperref}
\hypersetup{%
  colorlinks=true,
  linkcolor=blue,
  citecolor=blue,
  urlcolor=blue,
  linkbordercolor={0 0 1}
}
% -----------------------------------
\usepackage[style=authoryear-icomp,backend=biber]{biblatex}
\addbibresource{citation.bib}
% -----------------------------------
\usepackage{fancyhdr}
\newcommand\course{MATH-UA.0252\\Numerical Analysis}
\newcommand\hwnumber{4-2}                  % <-- homework number
\newcommand\NetIDa{Ryan Sh\`iji\'e D\`u} 
\newcommand\NetIDb{October 6th, 2023}
\pagestyle{fancyplain}
\headheight 35pt
\lhead{\NetIDa\\\NetIDb}
\chead{\textbf{\Large Worksheet \hwnumber}}
\rhead{\course}
\lfoot{}
\cfoot{}
\rfoot{\small\thepage}
\headsep 1.5em
% -----------------------------------
\usepackage{titlesec}
\renewcommand\thesubsection{(\arabic{section}.\alph{subsection})}
\titleformat{\subsection}[runin]
        {\normalfont\bfseries}
        {\thesubsection}% the label and number
        {0.5em}% space between label/number and subsection title
        {}% formatting commands applied just to subsection title
        []% punctuation or other commands following subsection title
% -----------------------------------
\setlength{\parindent}{0.0in}
\setlength{\parskip}{0.1in}
% -----------------------------------
\newcommand{\de}{\mathrm{d}}
\newcommand{\DD}{\mathrm{D}}
\newcommand{\pe}{\partial}
\newcommand{\mcal}{\mathcal}
%\newcommand{\pdx}{\left|\frac{\partial}{\partial_x}\right|}

\newcommand{\dsp}{\displaystyle}

\newcommand{\norm}[1]{\left\Vert #1 \right\Vert}
%\newcommand{\mean}[1]{\left\langle #1 \right\rangle}
\newcommand{\mean}[1]{\overline{#1}}
\newcommand{\inner}[2]{\left\langle #1,#2\right\rangle}

\newcommand{\ve}[1]{\boldsymbol{#1}}

\newcommand{\thus}{\Rightarrow \quad }
\newcommand{\fff}{\iff\quad}
\newcommand{\qdt}[1]{\quad \mbox{#1} \quad}

\renewcommand{\Re}{\mathrm{Re}}
\renewcommand{\Im}{\mathrm{Im}}
\newcommand{\E}{\mathbb{E}}
\newcommand{\lap} {\nabla^2}
\renewcommand{\div}{\nabla\cdot}

\newcommand{\csch}{\text{csch}}
\newcommand{\sech}{\text{sech}}


\newcommand{\hot}{\text{h.o.t.}}

\newcommand{\ssp}{\left.\qquad\right.}

\newcommand{\var}{\text{var}}
\newcommand{\cov}{\text{cov}}

%%%%%%%%%%%%%%%%%%%%%%%%%%%%%%%%%%%%%%%%%%%%%%%%%%
\makeatletter
\newcommand*{\mint}[1]{%
  % #1: overlay symbol
  \mint@l{#1}{}%
}
\newcommand*{\mint@l}[2]{%
  % #1: overlay symbol
  % #2: limits
  \@ifnextchar\limits{%
    \mint@l{#1}%
  }{%
    \@ifnextchar\nolimits{%
      \mint@l{#1}%
    }{%
      \@ifnextchar\displaylimits{%
        \mint@l{#1}%
      }{%
        \mint@s{#2}{#1}%
      }%
    }%
  }%
}
\newcommand*{\mint@s}[2]{%
  % #1: limits
  % #2: overlay symbol
  \@ifnextchar_{%
    \mint@sub{#1}{#2}%
  }{%
    \@ifnextchar^{%
      \mint@sup{#1}{#2}%
    }{%
      \mint@{#1}{#2}{}{}%
    }%
  }%
}
\def\mint@sub#1#2_#3{%
  \@ifnextchar^{%
    \mint@sub@sup{#1}{#2}{#3}%
  }{%
    \mint@{#1}{#2}{#3}{}%
  }%
}
\def\mint@sup#1#2^#3{%
  \@ifnextchar_{%
    \mint@sup@sub{#1}{#2}{#3}%
  }{%
    \mint@{#1}{#2}{}{#3}%
  }%
}
\def\mint@sub@sup#1#2#3^#4{%
  \mint@{#1}{#2}{#3}{#4}%
}
\def\mint@sup@sub#1#2#3_#4{%
  \mint@{#1}{#2}{#4}{#3}%
}
\newcommand*{\mint@}[4]{%
  % #1: \limits, \nolimits, \displaylimits
  % #2: overlay symbol: -, =, ...
  % #3: subscript
  % #4: superscript
  \mathop{}%
  \mkern-\thinmuskip
  \mathchoice{%
    \mint@@{#1}{#2}{#3}{#4}%
        \displaystyle\textstyle\scriptstyle
  }{%
    \mint@@{#1}{#2}{#3}{#4}%
        \textstyle\scriptstyle\scriptstyle
  }{%
    \mint@@{#1}{#2}{#3}{#4}%
        \scriptstyle\scriptscriptstyle\scriptscriptstyle
  }{%
    \mint@@{#1}{#2}{#3}{#4}%
        \scriptscriptstyle\scriptscriptstyle\scriptscriptstyle
  }%
  \mkern-\thinmuskip
  \int#1%
  \ifx\\#3\\\else_{#3}\fi
  \ifx\\#4\\\else^{#4}\fi  
}
\newcommand*{\mint@@}[7]{%
  % #1: limits
  % #2: overlay symbol
  % #3: subscript
  % #4: superscript
  % #5: math style
  % #6: math style for overlay symbol
  % #7: math style for subscript/superscript
  \begingroup
    \sbox0{$#5\int\m@th$}%
    \sbox2{$#5\int_{}\m@th$}%
    \dimen2=\wd0 %
    % => \dimen2 = width of \int
    \let\mint@limits=#1\relax
    \ifx\mint@limits\relax
      \sbox4{$#5\int_{\kern1sp}^{\kern1sp}\m@th$}%
      \ifdim\wd4>\wd2 %
        \let\mint@limits=\nolimits
      \else
        \let\mint@limits=\limits
      \fi
    \fi
    \ifx\mint@limits\displaylimits
      \ifx#5\displaystyle
        \let\mint@limits=\limits
      \fi
    \fi
    \ifx\mint@limits\limits
      \sbox0{$#7#3\m@th$}%
      \sbox2{$#7#4\m@th$}%
      \ifdim\wd0>\dimen2 %
        \dimen2=\wd0 %
      \fi
      \ifdim\wd2>\dimen2 %
        \dimen2=\wd2 %
      \fi
    \fi
    \rlap{%
      $#5%
        \vcenter{%
          \hbox to\dimen2{%
            \hss
            $#6{#2}\m@th$%
            \hss
          }%
        }%
      $%
    }%
  \endgroup
}

\begin{document} 

\section{Norms Equivalency}
Two norms in a finite-dimensional linear space $X$ (e.g.: $\mathbb{R}^n$), $\norm{\;\cdot\;}_a$ and $\norm{\;\cdot\;}_b$, are called equivalent if there is a constant $c$ such that for all $x$ in $X$,
\begin{align}
    \norm{\ve x}_a\leq c\norm{\ve x}_b,\qquad \norm{\ve x}_b\leq c\norm{\ve x}_a.\label{eq:norm_equiv}
\end{align}

\subsection{}
Suppose $\norm{\;\cdot\;}_a$ and $\norm{\;\cdot\;}_b$ are equivalent, and we know that an algorithm produces a sequence of vectors $\{\ve e_n\}_{n\geq 1}$, $\norm{\ve e_n}_a\to 0$ as $n\to\infty$. What could we conclude about $\norm{\ve e_n}_b$'s behavior for $n\to\infty$?

\subsection{}
We first show that the vector norms on $\mathbb{R}^n$, $\norm{\;\cdot\;}_{2}$ and $\norm{\;\cdot\;}_{\infty}$, are equivalent. To do this prove the inequality:
\begin{align*}
    \norm{\ve x}_\infty \leq \norm{\ve x}_2 \leq \sqrt{n}\norm{\ve x}_\infty.
\end{align*}

\subsection{}
The induced matrix norm on $\mathbb{R}^{n\times n}$: $\norm{\;\cdot\;}_{2}$ and $\norm{\;\cdot\;}_{\infty}$ are equivalent as well. Prove the inequality
\begin{align*}
    &\norm{\ve A}_\infty \leq \sqrt{n}\norm{\ve A}_2,\\
    &\norm{\ve A}_2 \leq \sqrt{n}\norm{\ve A}_\infty.
\end{align*}

\subsection{}
(Challenge) Prove that: in a finite-dimensional linear space, all norms are equivalent; that is, any two satisfy \eqref{eq:norm_equiv} with some $c$, depending on the pair of norms \parencite[p.217]{Lax_07}.

One inequality is relatively simple, the other one requires some big theorems from analysis. Read about the proof in Lax's book if you are interested.

\section{Condition numbers based on different norms}
\subsection{}
Let $A \in \mathbb{R}^{n\times n}$ be defined by
\begin{equation*}
    A = %
    \begin{bmatrix}
      1      & 0      & 0      & 0      &  0      \\ 
      1      & 1      & 0      & 0      &  0      \\
      1      & 0      & 1      & 0      &  0      \\
      \vdots & \vdots & \vdots & \ddots &  \vdots \\
      1      & 0      & 0      & 0      &  1
    \end{bmatrix}.
\end{equation*}
Calculate $\kappa_1(A)$ and $\kappa_\infty(A)$. We see that a matrix can be well or ill-conditioned depending on the choice of norms.

\subsection{}
Indeed, we solve $A\ve x = \ve b$ and $A(\ve x+\Delta \ve x) = (\ve b+\Delta \ve b)$ where
\begin{align*}
    \ve b = \begin{bmatrix}
    1\\1\\\vdots\\1
    \end{bmatrix},\qquad \ve x = \begin{bmatrix}
    1\\0\\\vdots\\0
    \end{bmatrix},\qdt{and} \Delta\ve b = \begin{bmatrix}
    \epsilon\\0\\\vdots\\0
    \end{bmatrix}
\end{align*}
Check that we have for both norms:
\begin{align*}
    \frac{\norm{\Delta\ve x}}{\norm{\ve x}} \leq \kappa(A)\frac{\norm{\Delta\ve b}}{\norm{\ve b}}.
\end{align*}

\section{Conditional number for the Hilbert matrix}  
The Hilbert matrix $H\in \mathbb R^{n\times n}$ is a matrix with
  entries
  $$
  h_{ij} = \frac{1}{i+j-1}.
  $$ 
\subsection{}
Using MATLAB or Python, compute the 2-norm-based condition
  numbers for $n=3,5,10,20,25$.  
  
\subsection{}
  Let's consider a relative right hand
  side perturbation $\delta\boldsymbol b$ of a linear system with
  $\|\delta\boldsymbol b\|_2/\|\boldsymbol b\|_2\approx
  10^{-15}$. Write down the corresponding bounds $\|\delta\boldsymbol
  x\|_2/\|\boldsymbol x\|_2$ from the theory we discussed in class.

\subsection{}
  Now, let's compute the actual error. Use the right-hand side vector
  with entries $b_i = \sum_{j=1}^n(j/(i+j-1))$ chosen such that the
  solution vector has entries $x_i=i$. Now, Compute the numerical
  solutions\footnote{Note that all these computations contain tiny
    errors due to the final precision of computer computations.}
  $\boldsymbol x$, then re-compute $\boldsymbol b=H\boldsymbol x$ and
  compare the relative right-hand side error and the relative error
  in the solutions. How much are these better than the estimates you
  got from the condition number? 
    
    
\printbibliography

\end{document}