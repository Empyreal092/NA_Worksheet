\documentclass[11pt,letterpaper]{article}
\usepackage[utf8]{inputenc}
\usepackage[left=1in,right=1in,top=1in,bottom=1in]{geometry}
\usepackage{amsfonts,amsmath}
\usepackage{graphicx,float}
% -----------------------------------
\usepackage{hyperref}
\hypersetup{%
  colorlinks=true,
  linkcolor=blue,
  citecolor=blue,
  urlcolor=blue,
  linkbordercolor={0 0 1}
}
% -----------------------------------
\usepackage{fancyhdr}
\newcommand\course{MATH-UA.0252\\Numerical Analysis}
\newcommand\hwnumber{11}                  % <-- homework number
\newcommand\NetIDa{Ryan Sh\`iji\'e D\`u} 
\newcommand\NetIDb{December 2nd, 2022}
\pagestyle{fancyplain}
\headheight 35pt
\lhead{\NetIDa\\\NetIDb}
\chead{\textbf{\Large Worksheet \hwnumber}}
\rhead{\course}
\lfoot{}
\cfoot{}
\rfoot{\small\thepage}
\headsep 1.5em
% -----------------------------------
\usepackage{titlesec}
\renewcommand\thesubsection{(\arabic{section}.\alph{subsection})}
\titleformat{\subsection}[runin]
        {\normalfont\bfseries}
        {\thesubsection}% the label and number
        {0.5em}% space between label/number and subsection title
        {}% formatting commands applied just to subsection title
        []% punctuation or other commands following subsection title
% -----------------------------------
\setlength{\parindent}{0.0in}
\setlength{\parskip}{0.1in}
% -----------------------------------
\newcommand{\de}{\mathrm{d}}
\newcommand{\DD}{\mathrm{D}}
\newcommand{\pe}{\partial}
\newcommand{\mcal}{\mathcal}
%\newcommand{\pdx}{\left|\frac{\partial}{\partial_x}\right|}

\newcommand{\dsp}{\displaystyle}

\newcommand{\norm}[1]{\left\Vert #1 \right\Vert}
%\newcommand{\mean}[1]{\left\langle #1 \right\rangle}
\newcommand{\mean}[1]{\overline{#1}}
\newcommand{\inner}[2]{\left\langle #1,#2\right\rangle}

\newcommand{\ve}[1]{\boldsymbol{#1}}

\newcommand{\thus}{\Rightarrow \quad }
\newcommand{\fff}{\iff\quad}
\newcommand{\qdt}[1]{\quad \mbox{#1} \quad}

\renewcommand{\Re}{\mathrm{Re}}
\renewcommand{\Im}{\mathrm{Im}}
\newcommand{\E}{\mathbb{E}}
\newcommand{\lap} {\nabla^2}
\renewcommand{\div}{\nabla\cdot}

\newcommand{\csch}{\text{csch}}
\newcommand{\sech}{\text{sech}}


\newcommand{\hot}{\text{h.o.t.}}

\newcommand{\ssp}{\left.\qquad\right.}

\newcommand{\var}{\text{var}}
\newcommand{\cov}{\text{cov}}

%%%%%%%%%%%%%%%%%%%%%%%%%%%%%%%%%%%%%%%%%%%%%%%%%%
\makeatletter
\newcommand*{\mint}[1]{%
  % #1: overlay symbol
  \mint@l{#1}{}%
}
\newcommand*{\mint@l}[2]{%
  % #1: overlay symbol
  % #2: limits
  \@ifnextchar\limits{%
    \mint@l{#1}%
  }{%
    \@ifnextchar\nolimits{%
      \mint@l{#1}%
    }{%
      \@ifnextchar\displaylimits{%
        \mint@l{#1}%
      }{%
        \mint@s{#2}{#1}%
      }%
    }%
  }%
}
\newcommand*{\mint@s}[2]{%
  % #1: limits
  % #2: overlay symbol
  \@ifnextchar_{%
    \mint@sub{#1}{#2}%
  }{%
    \@ifnextchar^{%
      \mint@sup{#1}{#2}%
    }{%
      \mint@{#1}{#2}{}{}%
    }%
  }%
}
\def\mint@sub#1#2_#3{%
  \@ifnextchar^{%
    \mint@sub@sup{#1}{#2}{#3}%
  }{%
    \mint@{#1}{#2}{#3}{}%
  }%
}
\def\mint@sup#1#2^#3{%
  \@ifnextchar_{%
    \mint@sup@sub{#1}{#2}{#3}%
  }{%
    \mint@{#1}{#2}{}{#3}%
  }%
}
\def\mint@sub@sup#1#2#3^#4{%
  \mint@{#1}{#2}{#3}{#4}%
}
\def\mint@sup@sub#1#2#3_#4{%
  \mint@{#1}{#2}{#4}{#3}%
}
\newcommand*{\mint@}[4]{%
  % #1: \limits, \nolimits, \displaylimits
  % #2: overlay symbol: -, =, ...
  % #3: subscript
  % #4: superscript
  \mathop{}%
  \mkern-\thinmuskip
  \mathchoice{%
    \mint@@{#1}{#2}{#3}{#4}%
        \displaystyle\textstyle\scriptstyle
  }{%
    \mint@@{#1}{#2}{#3}{#4}%
        \textstyle\scriptstyle\scriptstyle
  }{%
    \mint@@{#1}{#2}{#3}{#4}%
        \scriptstyle\scriptscriptstyle\scriptscriptstyle
  }{%
    \mint@@{#1}{#2}{#3}{#4}%
        \scriptscriptstyle\scriptscriptstyle\scriptscriptstyle
  }%
  \mkern-\thinmuskip
  \int#1%
  \ifx\\#3\\\else_{#3}\fi
  \ifx\\#4\\\else^{#4}\fi  
}
\newcommand*{\mint@@}[7]{%
  % #1: limits
  % #2: overlay symbol
  % #3: subscript
  % #4: superscript
  % #5: math style
  % #6: math style for overlay symbol
  % #7: math style for subscript/superscript
  \begingroup
    \sbox0{$#5\int\m@th$}%
    \sbox2{$#5\int_{}\m@th$}%
    \dimen2=\wd0 %
    % => \dimen2 = width of \int
    \let\mint@limits=#1\relax
    \ifx\mint@limits\relax
      \sbox4{$#5\int_{\kern1sp}^{\kern1sp}\m@th$}%
      \ifdim\wd4>\wd2 %
        \let\mint@limits=\nolimits
      \else
        \let\mint@limits=\limits
      \fi
    \fi
    \ifx\mint@limits\displaylimits
      \ifx#5\displaystyle
        \let\mint@limits=\limits
      \fi
    \fi
    \ifx\mint@limits\limits
      \sbox0{$#7#3\m@th$}%
      \sbox2{$#7#4\m@th$}%
      \ifdim\wd0>\dimen2 %
        \dimen2=\wd0 %
      \fi
      \ifdim\wd2>\dimen2 %
        \dimen2=\wd2 %
      \fi
    \fi
    \rlap{%
      $#5%
        \vcenter{%
          \hbox to\dimen2{%
            \hss
            $#6{#2}\m@th$%
            \hss
          }%
        }%
      $%
    }%
  \endgroup
}

\begin{document}

\section{Polynomial interpolation basics}
True or False?
\begin{itemize}
\item For the nodes $x_0=0, x_1=1, x_2= 2$, the
  Lagrange interpolation polynomial $L_0(x)$ is $-x^2 + 1$.
\item We compute the Hermite interpolant with 3 distinct nodes of
  a function $f$ that is a polynomial of degree $4$. Then this
  Hermite interpolant is identical to $f$. (In short: Hermite
  interpolation with 3 nodes is exact for polynomials of degree 4.)
\item Hermite interpolation with 4 distinct nodes is exact for
  polynomials of degree 6.
\end{itemize}

\section{Lagrange interpolation polynomial example}
Let $x_0,\ldots, x_n$ be distinct interpolation nodes, and let
  $$
  p_n(x) = \sum^n_{k=0}L_{k}(x)(x_k)^j,
  $$
  where $j$ is an integer and $n \geq j>0$.
  What are the values of $p_n(0)$ and $p_n(1)$?

\section{Hermite interpolation polynomial example}
Recall that the Hermite interpolation of a function $f$ at the points $x_0,x_1,x_2$ has the form 
$$p(x) = \sum_{j=0}^2H_j(x)f(x_j)
+ \sum_{j=0}^2K_j(x)f'(x_j).$$ 
  
\subsection{}
Show that the polynomial
$$ -\frac{1}{\pi}x^2 + x$$ 
is the Hermite interpolation polynomial of $f(x):=\sin(x)$ based on the nodes $x_0=0$, $x_1=\pi$.
  
\subsection{}
Show that the polynomial $K_2(x)$ in
this representation for $x_0=0,x_1=1,x_2=2$ is given by
$$
\frac{1}{4}x^5 - x^4 + \frac{5}{4}x^3 - \frac{1}{2}x^2.
$$

\newpage
\section{Deriving a new quadrature rule}
Given $f:[0,1]\rightarrow \mathbb{R}$, you want to derive a new
  quadrature rule that does uses not only function values, but also
  gradient values:
\begin{align}
    \int_0^1 f(x)\,\de x \approx \alpha_0 f(0) + \alpha_1 f'(0) + \alpha_2 f(1). \label{eq:11_3a}
\end{align}
  
\subsection{}
First, find polynomials $J_0, J_1, J_2 \in \mathcal{P}_2$, with the
  following properties:
  \begin{align*}
    & J_0(0)=1,\quad J_0'(0)=0,\quad J_0(1)=0\\
    & J_1(0)=0,\quad J_1'(0)=1,\quad J_1(1)=0\\
    & J_2(0)=0,\quad J_2'(0)=0,\quad J_2(1)=1.
  \end{align*}
  ({\em Hint:} For each $J_i$, make an ansatz for a quadratic polynomial using the monomial basis.)

Given $f$, you can now define a polynomial approximation $p\in \mathcal{P}_2$ via 
\begin{align}
    p(x)=f(0)J_0(x)+f'(0)J_1(x)+f(1)J_2(x). \label{eq:11_3b}
\end{align}
The polynomial $p$ is an approximation to $f$ in the sense that
$p(0)=f(0)$, $p'(0)=f'(0)$ and $p(1)=f(1)$. 

\subsection{}
Use the polynomial
  $p$ derived in \eqref{eq:11_3b} and the same method used to derive the Newton-Cotes
  quadrature rules, to find the coefficients $\alpha_0$, $\alpha_1$ and
  $\alpha_2$ in \eqref{eq:11_3a}.
  
\subsection{}
Use your new quadrature rule to approximate
  $\dsp{\int_0^1\exp(2x)\sin^2(x)\,\de x}$, and also compare with
  Simpson's rule. The exact value of this integral is 1.2668\dots.


\section{Trapezoidal rule for smooth periodic functions}

We investigate how the (composite) trapezoidal rule performs for smooth, periodic functions. Consider integrating the smooth, periodic function $f(x) = e^{\sin x}$ over a single period. The exact value of the integral is
\[
I(f) = \int_0^{2\pi} e^{\sin x} \, \de x = 7.95492652101284527\dots.
\]

\subsection{}
Write down the composite trapezoidal rule $T_N(f)$ on equispaced nodes $0=x_0 \le  \dots \le x_N = 2\pi$ for estimating the value of this integral. 

\subsection{}
Simplify your expression for $T_N(f)$ using the periodicity of $f$.

\subsection{}
Show that $T_N(f)$ is equivalent to both a left-endpoint Riemann sum and a right-endpoint Riemann sum approximation to $I(f)$.

\subsection{}
Compute $T_N(f)$ for various progressively larger $N$. Plot the quadrature errors against $N$ on (i) a log-log plot, and (ii) a \texttt{semilogy} plot. What is the order of accuracy of the trapezoidal rule for smooth, periodic functions?


% \vfill
% \bibliographystyle{alpha}
% \bibliography{citation}

\end{document}